\section{Capital stock}
\label{sec:kstock}

When moving to flexible time
steps, cumulative investment over intermediate years needs to be accounted for.
One simple way to do this is to assume a constant growth rate for investment
between a past year and the current year:

\[
I_t = \left(1+ g^I_t \right)^n  I_{t-n}
\]

After some algebra, this leads to a motion equation for the capital stock that
takes the following form:\footnote{It is easy to see that the equation collapses
to the one-period motion equation when $n$ is equal to 1.}

\[
K^s_t =
   \left( 1- \delta_t \right)^n K^s_{t-n}
+  \frac {\left( 1 + g^I_t \right)^n - \left( 1 - \delta_t \right)^n}
         {g^I_t + \delta_t} I_{t-n}
\]

The resulting motion equation is no longer pre-determined as the contemporaneous
level of investment is needed to measure the annual growth in investment. The
equation is also quite sensitive to the relative values of $g^I$ and $\delta$.
In the model implementation, a somewhat transformed version of the motion
equation is used that includes an additional equation that captures investment
growth. Equation~(\ref{eq:invgfact}) is the investment growth factor. The value
of the expression is equal to the inverse of the annual rate of growth of
investment plus the depreciation rate. For example, when both are at 5~percent,
$\Psi$ takes the value 10. Equation~(\ref{eq:kst}) is the capital stock
accumulation function for multi-period time steps. Both formulas depend on the
contemporaneous level of investment. Equation~(\ref{eq:tks}) defines the
normalized capital stock. It is a fixed scalar of the non-normalized level where
the scaling factor is initialized using base year data,
i.e. $\chi^k_r = \mathit{TK^s}_{r,0}/K^s_{r,0}$

\begin{equation}
\label{eq:invgfact}
\Psi_{r,t} =
   \left[ \left( \frac {\mathit{XFD_{r,\mathit{inv},t}}}
      {\mathit{XFD_{r,\mathit{inv},t-n}}}\right)^{1/n} -1 + \delta_{r,t}
   \right]^{-1}
\end{equation}

\begin{equation}
\label{eq:kst}
\mathit{K^s_{r,t}} =
   \left[ \mathit{K^s_{r,t-n}}
-  \Psi_{r,t} \mathit{XFD_{r,\mathit{inv},t-n}}\right]
   \left( 1 - \delta_{r,t} \right)^n
+  \Psi_{r,t} \mathit{XFD_{r,\mathit{inv},t}}
\end{equation}

At times it is useful to measure sectoral investment needs. For a single time
step, this is relatively straightforward using the following expression:

\[
I_{r,a,t-1} =
   \frac{1}{\chi^k_r} \left[\sum_v{K^v_{r,a,v,t}}
-  (1-\delta_{r,t}) \sum_v{K^v_{r,a,v,t-1}} \right]
\]

\noindent where it is assumed that the factor $\chi^k$ applies uniformly across
all activities and not just for the aggregate capital stock. The value of
investment is equal to the volume times the price of investment given by the
variable $\mathit{PFD}$. In a multi-time step simulation, the formula can be
generalized to the following:

\[
I_{r,a,t-1} =
   \frac{1}{\chi^k_r} \frac {\delta_{r,t}}{1-(1-\delta_{r,t})^n}
      \left[\sum_v{K^v_{r,a,v,t}} - (1-\delta_{r,t})^n \sum_v{K^v_{r,a,v,t-n}}
      \right]
\]

\noindent under the assumption of equal investment in each year. Cumulative
investment is then $n$ times the annual level. The formula is derived from the
following:

\[
\sum_v{K^v_{r,a,v,t}} =
   (1-\delta_{r,t})^n \sum_v{K^v_{r,a,v,t-n}}
+  \chi^k_r \sum_{i=0}^{n-1}{(1-\delta_{r,t})^{n-i-1}I_{r,a,t-n+i}}
\]

\noindent and under the assumption that investment is equal in each year.

\section{Climate module}

Some of the key changes to the climate module are described
here. First, equation~(\ref{eq:cume}) is modified to account
for intra-period emissions. It simply multiplies the previous period's 
emissions by the step size.
Other strategies are available, including using the average
of emissions between two solution periods.

\begin{equation}
\label{eq:cumemstep}
\mathit{CumEmiInd}_t = \mathit{CumEmiInd}_{t-1} 
+ n (12/44) \left(\mathit{EmiGbl}_{\mathit{co2},t-1}
 + \mathit{EmiX}^C_{t-1} \right)
\end{equation}

Second, the one-step carbon cycle dynamics is converted to an n-step
equation, (\ref{eq:rn}). With a little bit of algebra, this
can be converted to the final form, equation~(\ref{eq:r}).

A few additional notes:

\begin{itemize}
\item It is easy to see that equation~(\ref{eq:r}) collapses to equation~(\ref{eq:r1})
when $n$ is equal  to 1.
\item 
The multi-step formulas explicitly assume that the
inter-period emissions are equal to $E_{t-1}$ for each year. One could make
an alternative assumption, such as taking the mean, i.e. $0.5(\mathit{Emi}^C_t + \mathit{Emi}^C_{t-1})$.
A more sophisticated formula would incorporate the inter-period growth in emissions,
$(\mathit{Emi}^C_t/\mathit{Emi}^C_{t-1})^{1/n}$, which could be integrated into equation~(\ref{eq:r}). With
relatively small time steps, this is unlikely to lead to very significant changes.
\end{itemize}

\begin{equation}
\label{eq:rn}
\mathit{R}_{b,t} = \left(1 - \displaystyle \frac {1}{\alpha \tau_b} \right)^{n} {\mathit{R}_{b,t-n}} + \frac{\phi_b}{44/12} \mathit{Emi}^C_{t-1} \sum_{i=0}^{n-1}{\left(1 - \displaystyle \frac {1}{\alpha \tau_b} \right)^{n-1-i}}
\end{equation}

\begin{equation}
\label{eq:r}
\mathit{R}_{b,t} =  \left(1 - \displaystyle \frac {1}{\alpha_{t-n} \tau_b} \right)^{n} {\mathit{R}_{b,t-n}}
+ {\alpha_{t-n} \tau_b} \frac{\phi_b}{44/12} \mathit{Emi}^C_{t-n} 
\left[1 - 
\left(1 - \displaystyle \frac {1}{\alpha_{t-n} \tau_b} \right)^{n} 
\right]
\end{equation}

The EBM model, which also has lags, is solved in annual time steps as described
in section~\ref{sec:climModule}.

\section{Resource depletion}

The resource depletion module needs a few modifications when 
the time step is greater than 1 year---in particular, inter-period
resource flows need to be taken into account as is the case for capital/investment
stock/flow consistency.

Equation~(\ref{eq:pRatio}) needs to reflect the price trends over multiple
periods (though the $\mathit{PTrend}$ variable is assumed to reflect an annual rate of change between the relevant time periods):

\begin{equation}
\label{eq:pRatiodyn}
\mathit{PRatio}_{r,a,t} = \left[ \frac{\mathit{PX}_{r,a,t}/\mathit{PGDP}_{r,t}} {\mathit{PX}_{r,a,t-n}/\mathit{PGDP}_{r,t-n}} \right]^{1/n} \bigg / \mathit{PTrend}_{r,a,t} 
\end{equation}

Equation~(\ref{eq:Res}), a stock equation, needs to take into account
the cumulative resource extraction flows between solution periods. These
latter are calculated assuming constant growth in extraction between 
periods. The cumulative
extraction between $t-n$ and $t$ is given by:

\[
\mathit{CumExtr}_t = \sum_{i=0}^{n-1}{\mathit{Extr}_{t-n+i}}
\]

\noindent If we assume constant growth between periods, we have:

\[
\mathit{Extr}_{t-n+i} = \mathit{Extr}_{t-n}\left(1 + g\right)^i
\]

\noindent where

\[
g = \left( \frac{\mathit{Extr}_{t}}{\mathit{Extr}_{t-n}} \right)^{1/n} - 1
\]

\noindent and thus

\[
\mathit{CumExtr}_t = \mathit{Extr}_{t-n} \displaystyle \sum_{i=0}^{n-1}{(1+g)^i} 
= \mathit{Extr}_{t-n} \displaystyle \frac{\left( 1+g \right)^n - 1} {g}
\]

\noindent Note that if the growth rate is zero, the growth
factor is replaced with the time step, i.e. $n$. With substitution, equation~(\ref{eq:cumext}) provides the final
expression for cumulative extraction.

\begin{equation}
\label{eq:cumext}
\mathit{CumExtr}_{r,a,t} = \frac{\mathit{Extr}_{t} - \mathit{Extr}_{t-n}}
{\displaystyle \left(\frac{\mathit{Extr}_{t}}{\mathit{Extr}_{t-n}}\right)^{1/n} - 1}
\end{equation}

Equation~(\ref{eq:Res}), the level of proven reserves, is replaced with equation~(\ref{eq:Resdyn}). It takes
into account cumulative extraction, as described above, and also
cumulative new discoveries.

\begin{equation}
\label{eq:Resdyn}
\mathit{Res}_{r,a,t} = \mathit{Res}_{r,a,t-n} - \mathit{CumExtr}_{r,a,t} 
+ \left( 1 - (1 - \xi_{r,a,t})^n \right) \mathit{YTD}_{r,a,t-n}
\end{equation}

The multi-step expression for the reserve profile equation, equation~(\ref{eq:ResP}), is provided in equation~(\ref{eq:ResPdyn}).
The one-year time step equation is nearly identical to the capital
stock equation from above from which we can deduce the final
closed form.

\begin{equation}
\label{eq:ResPdyn}
\mathit{Res}^p_{r,a,t} = \left(1 - \rho^x_{r,a,t} \right)^n \mathit{Res}^p_{r,a,t-n} + \xi_{r,a,t} \mathit{YTD}_{r,a,t-n}
\frac{\left(1 - \rho^x_{r,a,t} \right)^n - \left(1 - \xi_{r,a,t} \right)^n }{\xi_{r,a,t} - \rho^x_{r,a,t}}
\end{equation}

The motion equation for unproven reserves, equation~(\ref{eq:YTD}),
is replaced with equation~(\ref{eq:YTDdyn}).

\begin{equation}
\label{eq:YTDdyn}
\mathit{YTD}_{r,a,t} = \left( 1 - \xi_{r,a,t-1} \right)^n \mathit{YTD}_{r,a,t-n}
\end{equation}

The potential supply equation, equation~(\ref{eq:xfPotdyn}), is nearly
identical except that the resource gap is divided by the time step.

\begin{equation}
\label{eq:xfPotdyn}
\mathit{XF}^{\mathit{Pot}}_{r,a,t} = \rho^x_t \mathit{Res}^p_{r,a,t} + \mathit{ResGap}_{r,a,t}/n
\end{equation}

\section{Knowledge stock}
\label{sec:knowledge}

The starting point for the model implementation is the 1-step
motion equation for the knowledge stock\footnote{In the 
model the variable $\mathit{RD}$ is given by 
$\mathit{XFD}_{\mathit{rd}}$.}:

\[
\mathit{KN}_{t} = \left(1-\delta \right) \mathit{KN}_{t-1}
+ \sum_{k=0,i=t}^{N,t-N}{\gamma_k \mathit{RD}_{i}} = \left(1-\delta \right) \mathit{KN}_{t-1} + G'\mathit{RDL}_{t,t}
\]

\noindent where the last term is the summation expression in vector
form (the double-indexed subscript will become clearer below). $G'\mathit{RDL}$
represents the gamma-weighted sum of the lags of R\&D expenditures.
From this, we can use induction and convert the one-step expression to
an $n$-step expression, equation~(\ref{eq:knmod}). The first term
on the right-hand side is intuitive. The second term adds all
of the relevant cumulative weighted-lags and includes the
appropriate depreciation for the intermediate years. Say for example
we are evaluating knowledge stock in year 2025 with respect to 2020. The
sum will be over all years between 2021 and 2025 inclusive.
Equation~(\ref{eq:rdl}) calculates the cumulative weighted
lags for each year between $t-n+1$ and $t$, e.g. between
2021 and 2025 for the example above. It does require, nonetheless, the level of R\&D 
expenditures for the intermediate years. Since these have differing weights
over time, we need to keep track of these explicitly.
In the GAMS code, most equations are indexed over $t$,
which has variable step sizes. The R\&D variable is
calculated for each year, assuming constant growth
in the intermediate years. Equation~(\ref{eq:rdgr})
determines the multi-step growth rate. Equation~(\ref{eq:rd1})
defines R\&D expenditures for each year, irrespective
of the model's time step.

\begin{equation}
\label{eq:knmod}
\mathit{KN}_{t} = \left( 1 - \delta \right)^n \mathit{KN}_{t-n}
+ \sum_{i=t-n+1}^{t}{\left( 1 - \delta \right)^{t-i} \mathit{RDL}_{i,t}}
\end{equation}

\begin{equation}
\label{eq:rdl}
\mathit{RDL}_{i,t} = \sum_{k=0, j=i-k}^{N,i-N}{\gamma_k \mathit{RD}_j}
\qquad \textrm{for } t-n<i \le t
\end{equation}

\begin{equation}
\label{eq:rdgr}
\mathit{RD}_{t} = \left( 1 + \mathit{gr}_t \right)^{n}
\mathit{RD}_{t-n}
\end{equation}

\begin{equation}
\label{eq:rd1}
\mathit{RD}_{i} = \left( 1 + \mathit{gr}_t \right)^{n-(t-i)}
\mathit{RD}_{t-n} \qquad \textrm{for } t-n<i<t
\end{equation}

In the model implementation, a macro is used to substitute
equation~(\ref{eq:rdl}) into equation~(\ref{eq:knmod}).
And equations~(\ref{eq:rdgr}) and~(\ref{eq:rd1}) can be collapsed to equation~(\ref{eq:rd})
thereby eliminating the growth rate of R\&D expenditures, where
the calculations are done for each year between $t-n$ and $t$.

\begin{equation}
\label{eq:rd}
\mathit{RD}_{i} = \mathit{RD}_{t-n}^{(t-i)/n}
\mathit{RD}_{t}^{(n-(t-i))/n} \qquad \textrm{for } t-n<i<t
\end{equation}

If R\&D is growing at a steady rate of $g$, we can then write:

\begin{equation}
\label{eq:knss}
\begin{array} {l c l}
\mathit{KN}_{t} & = & \displaystyle \left(1-\delta \right) \mathit{KN}_{t-1}
+ \sum_{k=0}^N{\gamma_{k} \mathit{RD}_{t-N} \left(1+g\right)^{N-k}} \\
{}  & = & \displaystyle \left(1-\delta \right) \mathit{KN}_{t-1}
+ \mathit{RD}_{t} \sum_{k=0}^N{ \frac{\gamma_{k}}{\left(1+g\right)^k} } \\
\end{array}
\end{equation}

\noindent In steady state, the summation term is a constant and we can
thus write:

\[
\mathit{KN}_{t} = \displaystyle \left(1-\delta \right) \mathit{KN}_{t-1}
+ \beta \mathit{RD}_{t} \qquad \textrm{where } \beta = \displaystyle \sum_{k=0}^N{ \frac{\gamma_{k}}{\left(1+g\right)^k} }
\]

\noindent The formula can also be converted to work with multi-year time steps.

\[
\begin{array}{l c l}
\mathit{KN}_{t}
& = & \displaystyle \left(1-\delta \right) \mathit{KN}_{t-1} + \beta \mathit{RD}_{t} \\
{} & = & \displaystyle \left(1-\delta \right) \left[\left(1-\delta \right) \mathit{KN}_{t-2} + \beta \mathit{RD}_{t-1} \right] + \beta \mathit{RD}_{t} \\
{} & = & \displaystyle \left(1-\delta \right)^2 \mathit{KN}_{t-2} + \left(1-\delta \right) \beta \mathit{RD}_{t-1} + \beta \mathit{RD}_{t} \\
{} & = & \displaystyle \left(1-\delta \right)^3 \mathit{KN}_{t-3} + \left(1-\delta \right)^2 \beta \mathit{RD}_{t-2} + \left(1-\delta \right) \beta \mathit{RD}_{t-1} + \beta \mathit{RD}_{t} \\
{} & \vdots & {} \\
{} & = & \displaystyle \left(1-\delta \right)^n \mathit{KN}_{t-n} + \beta \sum_{i=0}^{n-1}{\left(1-\delta \right)^i  \mathit{RD}_{t-i}} \\
\end{array}
\]\

\noindent Assuming a constant growth rate for R\&D expenditures, the last expression becomes:
\[
\mathit{KN}_{t} = \displaystyle \left(1-\delta \right)^n \mathit{KN}_{t-n} + \beta \sum_{i=0}^{n-1}{\left(1-\delta \right)^i  \mathit{RD}_{t} \left(1+g\right)^{-i}}
\]

\noindent where
\[
\mathit{RD}_{t} = \mathit{RD}_{t-i}  \left(1+g\right)^{i}
\]

\noindent After a bit of algebra, it is straightforward to show that the summation expression simplifies to:

\[
\sum_{i=0}^{n-1}{\frac{\left(1-\delta \right)^i} {\left(1+g\right)^{i}}} = \frac{1}{\left( 1 + g\right)^{n-1}}
\frac{\left(1+g\right)^n - \left(1-\delta\right)^n}{g + \delta}
\]

\noindent The final expression can thus be written as:
\[
\begin{array}{l c l}
\mathit{KN}_{t} & = & \displaystyle \left(1-\delta \right)^n \mathit{KN}_{t-n} + \beta \frac{\mathit{RD}_{t}}{\left( 1 + g\right)^{n-1}} \frac{\left(1+g\right)^n - \left(1-\delta\right)^n}{g + \delta} \\
& = & \displaystyle \left(1-\delta \right)^n \mathit{KN}_{t-n} + {\beta\left( 1 + g\right)\mathit{RD}_{t-n}} \frac{\left(1+g\right)^n - \left(1-\delta\right)^n}{g + \delta} \\
\end{array}
\]

These formulas rely on $n > N$, or that that the growth rate is uniform back through period $t-N$.

From the last formula, we can show that in steady-state the following
relation must hold for the ratio of knowledge stocks relative to R\&D expenditures:

\[
\frac{\mathit{KN}} {\mathit{RD}} = \beta \frac{1+g}{g+\delta}
\]

\subsection*{Implementation}

Initializing the stock of knowledge and the pre-base year R\&D expenditures
is an issue. The strategy followed so far is based on the following steps:

\begin{itemize}
	\item {Assume a given base year level of R\&D. Calculate
		the knowledge stock ($\mathit{KN}$) expenditure consistent with some assumption about the steady-state
		growth rate around the base year and the rate of knowledge depreciation.
		From the expression above, we have:
		\[{\mathit{KN}_{t_0}} = {\beta}  {\mathit{RD}_{t_0}}\frac{1+g}{g+\delta} =
		{\mathit{RD}_{t_0}} \frac{1+g}{g+\delta}\displaystyle \sum_{k=0}^N{ \frac{\gamma_{k}}{\left(1+g\right)^k} }
		\] Alternatively, one could make an assumption about the base
		year knowledge stock and invert the expression to calculate the
		R\&D expenditures consistent with the steady-state assumption.}
	\item {Back-cast the R\&D expenditures to the earliest period, that
		should be at least $N$ years back or more, for the largest $N$:
		\[
		\mathit{RD}_t = \mathit{RD}_{t+1} / \left(1+g\right) \qquad
		\textrm{for } t_f \le t < t_0-1
		\]
		where $t_f$ is the first year defined, e.g. 1960, and
		$t_0$ is the base year for the simulations, e.g. 2014.}
	\item {Back-cast the cumulative lag structure, $\mathit{RDL}$ using the following formula:
		\[
		\mathit{RDL}_{t,t_0}
		\sum_{k=0, i=t-k}^{N,t-N}{\gamma_{k}\mathit{RD}_i}
		\qquad \textrm{for } t_f+N \le t \le t_0
		\]}
	\item {Back-cast the stock of knowledge. This step is
		not strictly necessary.
		\[
		\mathit{KN}_t = \frac{\mathit{KN}_{t+1} - \mathit{RDL}_{t+1,t_0}}
		{1-\delta} \qquad \textrm{for } t_f+N-1 \le t < t_0
		\]}
	
\end{itemize}

\begin{figure}[ht]
	\centering
	\subfloat[][$\delta=0.7$, \scriptsize{$\lambda=0.9$, $N=50$, max=21.1, mean=25.8}]{
		\begin{tikzpicture}[scale=0.9]
		\begin{axis}[axis x line=bottom, axis y line=left, xlabel=Year,
		xtick={0,10,...,50}, ytick={0.2,0.4,...,1.0}, xmax=51, ymax=1.05,
		y tick label style={/pgf/number format/.cd,fixed,fixed zerofill,precision=1,/tikz/.cd}]
		\addplot[DodgerBlue2,very thick,mark=none,domain=0:50,samples=501]
		{0.006738536*exp(0.7*ln(\x+1)/(1-0.7) + \x*ln(0.9))};
		\addplot[lightgray,thin,mark=none]
		coordinates{(21.1,0) (21.1,0.995)} ;
		\addplot[Orchid,thin,dashed,mark=none]
		coordinates{(25.8,0) (25.8,1)} ;
		\end{axis}
		\end{tikzpicture}
	} \qquad
	\subfloat[][$\delta=0.7$, \scriptsize{$\lambda=0.8$, $N=35$, max=9.5, mean=13.4}]{
		\begin{tikzpicture}[scale=0.9]
		\begin{axis}[axis x line=bottom, axis y line=left, xlabel=Year,
		xtick={0,5,...,35}, ytick={0.2,0.4,...,1.0}, xmax=36, ymax=1.05,
		y tick label style={/pgf/number format/.cd,fixed,fixed zerofill,precision=1,/tikz/.cd}]
		\addplot[DodgerBlue2,very thick,mark=none,domain=0:35,samples=351]
		{0.03450336*exp(0.7*ln(\x+1)/(1-0.7) + \x*ln(0.8))};
		\addplot[lightgray,thin,mark=none]
		coordinates{(9.5,0) (9.5,0.995)} ;
		\addplot[Orchid,thin,dashed,mark=none]
		coordinates{(13.4,0) (13.4,1)} ;
		\end{axis}
		\end{tikzpicture}
	} \\
	\subfloat[][$\delta=0.6$, \scriptsize{$\lambda=0.85$, $N=25$, max=8.2, mean=11.6}]{
		\begin{tikzpicture}[scale=0.9]
		\begin{axis}[axis x line=bottom, axis y line=left, xlabel=Year,
		xtick={0,5,...,25}, ytick={0.2,0.4,...,1.0}, xmax=26, ymax=1.05,
		y tick label style={/pgf/number format/.cd,fixed,fixed zerofill,precision=1,/tikz/.cd}]
		\addplot[DodgerBlue2,very thick,mark=none,domain=0:25,samples=251]
		{0.135856271*exp(0.6*ln(\x+1)/(1-0.6) + \x*ln(0.85))};
		\addplot[lightgray,thin,mark=none]
		coordinates{(8.2,0) (8.2,0.995)} ;
		\addplot[Orchid,thin,dashed,mark=none]
		coordinates{(11.6,0) (11.6,1)} ;
		\end{axis}
		\end{tikzpicture}
	} \qquad
	\subfloat[][$\delta=0.4$, \scriptsize{$\lambda=0.8$, $N=15$, max=2.0, mean=5.4}]{
		\begin{tikzpicture}[scale=0.9]
		\begin{axis}[axis x line=bottom, axis y line=left, xlabel=Year,
		xtick={0,5,...,15}, ytick={0.2,0.4,...,1.0}, xmax=16, ymax=1.05,
		y tick label style={/pgf/number format/.cd,fixed,fixed zerofill,precision=1,/tikz/.cd}]
		\addplot[DodgerBlue2,very thick,mark=none,domain=0:15,samples=151]
		{0.751167359*exp(0.4*ln(\x+1)/(1-0.4) + \x*ln(0.8))};
		\addplot[lightgray,thin,mark=none]
		coordinates{(2.0,0) (2.0,0.995)} ;
		\addplot[Orchid,thin,dashed,mark=none]
		coordinates{(5.4,0) (5.4,1)} ;
		\end{axis}
		\end{tikzpicture}
	} \\
	\subfloat[][$\delta=0.7$, \scriptsize{$\lambda=0.9$, $N=25$, max=21.1, mean=16.3}]{
		\begin{tikzpicture}[scale=0.9]
		\begin{axis}[axis x line=bottom, axis y line=left, xlabel=Year,
		xtick={0,5,...,25}, ytick={0.2,0.4,...,1.0}, xmax=26, ymax=1.05,
		y tick label style={/pgf/number format/.cd,fixed,fixed zerofill,precision=1,/tikz/.cd}]
		\addplot[DodgerBlue2,very thick,mark=none,domain=0:25,samples=251]
		{0.006738536*exp(0.7*ln(\x+1)/(1-0.7) + \x*ln(0.9))};
		\addplot[lightgray,thin,mark=none]
		coordinates{(21.1,0) (21.1,0.99)} ;
		\addplot[Orchid,thin,dashed,mark=none]
		coordinates{(16.3,0) (16.3,1)} ;
		\end{axis}
		\end{tikzpicture}
	} \qquad
	\subfloat[][$\delta=0.5$, \scriptsize{$\lambda=0.8$, $N=15$, max=3.5, mean=6.3}]{
		\begin{tikzpicture}[scale=0.9]
		\begin{axis}[axis x line=bottom, axis y line=left, xlabel=Year,
		xtick={0,5,...,15}, ytick={0.2,0.4,...,1.0}, xmax=16, ymax=1.05,
		y tick label style={/pgf/number format/.cd,fixed,fixed zerofill,precision=1,/tikz/.cd}]
		\addplot[DodgerBlue2,very thick,mark=none,domain=0:15,samples=151]
		{0.48525365*exp(0.5*ln(\x+1)/(1-0.5) + \x*ln(0.8))};
		\addplot[lightgray,thin,mark=none]
		coordinates{(3.5,0) (3.5,0.995)} ;
		\addplot[Orchid,thin,dashed,mark=none]
		coordinates{(6.3,0) (6.3,1)} ;
		\end{axis}
		\end{tikzpicture}
	} \\
	\caption{Gamma distribution examples}
	\label{fig:gamDist}
\end{figure}
